\documentclass[a4paper]{exam}


\usepackage{amsfonts,amsmath,amsthm}
\usepackage[a4paper]{geometry}
\usepackage{hyperref}
\usepackage{xcolor}

\theoremstyle{definition}
\newtheorem{definition}{Definition}

\newcommand\Z{\ensuremath{\mathbb{Z}}}

\title{Problem Set 07: Proof Methods}
\author{CS/MATH 113 Discrete Mathematics}
\date{Spring 2024}

\boxedpoints

\printanswers

\begin{document}
\maketitle

For each question below, clearly write the statement to prove or disprove so that its logical structure is evident. Provide formal proofs. See \href{https://www.overleaf.com/learn/latex/Theorems_and_proofs#Proofs}{this guide} for typesetting proofs in \LaTeX.

If your proof is exceeding 10 lines, you are probably on the wrong track.

The following definitions may prove helpful when attempting the problems.

\begin{definition}[Prime and composite numbers]
A natural number (0, 1, 2, 3, 4, 5, 6, etc.) is called a \textit{prime number} (or a \textit{prime}) if it is greater than 1 and cannot be written as the product of two smaller natural numbers. The numbers greater than 1 that are not prime are called \textit{composite} numbers.  
\end{definition}

\begin{definition}[Even and odd numbers]
An \textit{even number} is an integer of the form $x=2k$ where $k$ is an integer; an \textit{odd number} is an integer of the form $x=2k+1$.  
\end{definition}

\begin{definition}[Parity]
The \textit{parity of a number} is its property of being even or odd.
\end{definition}

\begin{definition}[Rational number]
A \textit{rational number} can be written as $\frac{p}{q}$ where $p$ and $q$ are integers and $q\neq 0$. A number that is not rational is \textit{irrational}.
\end{definition}

\begin{questions}
  
\question
  Show through contraposition: If $x^2 - 6x + 5$ is even, then $x$ is odd.
  \begin{solution}
   
    Proof through contraposition.\\
    If $x$ is even then $x^2 - 6x + 5$ is odd.\\
    Any even number $x$ can be written as $x$ = 2$k$.\\
    Substituting $x$ with 2$k$,  $(2$k$)^2 - 6(2$k$) + 5$\\
     $4$k$^2 - 12$k$ + 5$\\
     $4$k$^2 - 12$k$ + 4 +1$\\
     $ 2(2$k$^2 - 6$k$ + 2) +1$\\
     $n=(2$k$^2 - 6$k$ + 2) $\\
     $ 2n +1$, this is by definition an odd number.\\
     Assuming $x$ to be even lead us to $x^2 - 6x + 5$ to be odd, then the opposite must also be true. Hence proved that If $x^2 - 6x + 5$ is even, then $x$ is odd.
  \end{solution}

\question Provide a counterexample to disprove: If $n$ is an integer and $n^2$ is divisible by 4, then $n$ is divisible by 4. Explain why it is a counterexample.

  \begin{solution}
   
    Taking n=6, then $n^2$=36.\\
    36 is divisible by 4, 36/4=9, however 6 is not divisible by 4 6/4= 1.5 which is not an integer. 
  \end{solution}
  
\question Prove using contradiction that $\sqrt{2}$ is irrational.

  \begin{solution}
   
    Assuming $\sqrt{2}$ is rational.\\
    Then it can be written as $\frac{p}{q}$ where $p$ and $q$ are integers and $q\neq 0$.\\
    $\sqrt{2}$ = $\frac{p}{q}$.\\
    squaring on both sides: 2= $\frac{p^2}{q^2}$\\
    Multiplying $q^2$ on both sides: $2q^2$= $p^2$\\
    Using the definition of even numbers, we know that $p^2$ is even, and if $p^2$ is even we can conclude that $p$ is also even. $p$ then can be written as $2n$.\\
    Substituting the value of $(2n)^2$= $2q^2$. \\
     $4n^2$= $2q^2$\\
     $2n^2$= $q^2$\\
     From the definition of even numbers, we can say that $q^2$ is an even number and if $q^2$ is an even number then q is also an even number.\\
     We have reached a contradiction, both $p$ and $q$ are even numbers which means they have 2 as a common factor where as if the numbers are rational then they only have 1 as a common factor. 
     Since there is a contradiction with the assumption, it must mean that  $\sqrt{2}$ is irrational.   
  \end{solution}
  
\fullwidth{For each of the following problems, clearly mention the proof method that you employ.}

\question Prove that for $n\in\Z$, $n$ is odd if and only if $5n + 6$ is odd.

  \begin{solution}
    
   (If n is odd then 5n + 6 is odd) and ( If 5n + 6 is odd then n is odd)\\
   If n is odd then 5n + 6 is odd. \\
   Using direct proof.\\
   n= $2k$ + 1.\\
   5($2k$ + 1) + 6.\\
   $10k$ + 11\\
   2( 5$k$ + 5) + 1 is odd.\\
   Hence proven that when n is odd then 5n + 6 is also odd.\\
   If 5n + 6 is odd then n is odd\\
   Using proof of contra position\\
    If n is even then 5n + 6 is even.\\
    If n is even then n can be written as, n = 2k.\\
    Replacing the value of n,  5($2k$) + 6  \\
    10$k$ + 6\\
    2(5$k$ + 3)\\
    (5$k$ + 3) = u\\
    Then 2$u$ which is even.\\
    Hence shown that both parts of statements are true so the bi conditional statement is true.
  \end{solution}

\question Prove or disprove: The sum of a rational and an irrational number is a rational number.

  \begin{solution}
    a= rational number x= irrational number b= rational number.\\
    a + x = b\\
    Using direct proof\\
    rearranging the statement, x= b - a\\
    since both b and a are rational numbers, their difference must also result in a rational number however x here is not rational.\\
    We can conclude that the statement is false.
    \end{solution}
  
\question Prove or disprove that for $(x^2 - y^2) \mod 4 \neq 2$ where $x$ and $y$ are integers.\\
  \textit{Hint}: a) Consider the different cases of parities of $x$ nd $y$. b) Use the method of \textit{proof by cases} and apply a proof \textit{without loss of generality} described in Section 1.8.2 in the book.

  \begin{solution}
   
    When $x$ is even and $y$ is odd, and x $\neq$ y.\\
    $x$ can be written as $2k$ and $y$ can be written as $2m$ + 1.\\
    $(2k)^2 - (2m + 1)^2$.\\
    $4k^2 - (4m^2 + 4m + 1)$\\
    $4k^2 -4m^2 - 4m - 1$\\
    $4(k^2 -m^2 - m) - 1$\\
    $4(k^2 -m^2 - m)$ mod 4 is 0 and -1 mod 4 =3.\\
    Hence the remainder in this case is 3.\\
     When $x$ is odd and $y$ is even, and x $\neq$ y.\\
      $x$ can be written as $2m$ + 1 and $y$ can be written as $2k$.\\
     $(2m + 1)^2$ -  $(2k)^2$\\
     $(4m^2 + 4m + 1)$ -$4k^2$\\
     $4m^2 + 4m -4k^2 + 1 $\\
    $4(-k^2 +m^2 + m) + 1$\\
    Remainder in this case is 1.\\
    When both x and y are even or both are odd then using proof without loss of generality.\\
    $x$ = 2$m$ and $y$ = $2n$\\
    $(2m)^2$ - $(2n)^2$\\
    $4m^2 - 4n^2$\\
    $4(m^2 - n^2)$\\
    Remainder in this case would be 0, the same would be true if both were odd numbers.\\
    Hence shown that  $(x^2 - y^2) \mod 4 \neq 2$ where $x$ and $y$ are integers.
    
    
  \end{solution}

\question 
  Prove or disprove that $2^n + 1$ is prime for every $n\in\Z^+$.

  \begin{solution}
    
    n=3\\
    $2^3 + 1$ = 8+1 =9\\
    9 is not a prime number so the statement is false.
  \end{solution}
\end{questions}
\end{document}
%%% Local Variables:
%%% mode: latex
%%% TeX-master: t
%%% End:
